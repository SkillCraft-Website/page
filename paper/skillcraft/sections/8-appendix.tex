\appendix

% (Optional) If you want appendix sections to be lettered:
% \renewcommand{\thesection}{\Alph{section}}

\section{Related Work}
% Existing tool-use benchmarks can be positioned along two main axes: whether the tools are genuinely executable with environment states close to production systems, and whether the tasks require long-horizon planning with composition across subtasks. On the more controlled end, BFCL~\citep{patilberkeley} abstracts tool invocation into structured function-parameter prediction, which makes it easy to compare models on calling format and tool selection. $\tau$-Bench~\citep{yao2024tau} and ACEBench~\citep{chen2025acebench} similarly emphasize multi-turn interaction and correctness of tool choice, typically using controlled settings to ensure reproducibility. Gorilla~\citep{patil2023gorilla} and AgentBench~\citep{liu2023agentbench} also primarily test whether an agent can pick the right API or tool, offering broad coverage across tools and domains. However, these benchmarks rarely force agents to perform programmatic abstraction of recurring structures, because many tasks can be solved with one-shot selection or short tool-call chains. ToolLLM~\citep{qin2023toolllm} further studies tool retrieval and model fine-tuning, but likewise focuses on single-use accuracy rather than cross-task reuse of patterns. For SkillCraft, , these works are useful baselines for whether a call is correct, but they provide weak signal for whether an agent consolidates recurring call patterns into reusable procedures, which is essential for explaining why some agents only succeed once while others become increasingly efficient on structurally similar subtasks.

% Moving closer to realistic execution, a growing set of benchmarks places agents in executable tools or high-fidelity application environments. AppWorld~\citep{trivedi2024appworld} supports richer interaction and state transitions through application-level environments. The MCP ecosystem (including MCPWorld~\citep{yan2025mcpworld}, MCP-RADAR~\citep{gao2025mcp}, MCPEval~\citep{liu2025mcpeval}, MCP-AgentBench~\citep{guo2025mcp}, LiveMCPBench~\citep{mo2025livemcpbench}, MCPAtlas~\citep{bandimcp}, etc.) standardizes tool integration and enables more systematic evaluation across servers and applications, but many tasks remain single-application and often rely on simplified or manually designed initial states. WebArena~\citep{zhou2023webarena}, OSWorld~\citep{xie2024osworld}, SWE-Bench~\citep{jimenez2024swebench} and TheAgentCompany~\citep{xu2024theagentcompanybenchmarkingllmagents} shift the emphasis toward long-horizon execution and error recovery in real web, desktop, and codebase workflows. GAIA~\citep{mialon2023gaia}, ARE~\citep{froger2025scaling}, and BrowseComp~\citep{wei2025browsecomp} focus on broad competence in daily tasks or browsing-based retrieval. Tool Decathlon (Toolathlon)~\citep{li2025tool} is pivotal on this trajectory: it combines real tools, fuzzy instructions, execution-based verification, and long-horizon cross-application workflows within one framework, setting a strong realism-oriented reference point. However, most benchmarks remain centered on the question of whether an agent can finish a task given atomic tools, meaning that even long tasks can often be solved by planning from scratch and incrementally invoking atomic tools. SkillCraft complements this landscape by organizing tasks into families with recurring substructure and scalable difficulty, so that skill accumulation becomes an observable and comparable capability rather than an incidental byproduct of long trajectories.

% On the agent pipeline side, mainstream tool-calling systems largely inherit the “reasoning–acting–observing" loop paradigm established by ReAct~\citep{yao2023react}, which interweaves natural language reasoning with external tool invocation, forming the foundational skeleton of today's tool-using agents. In parallel, another line of work treats programs or skills as the basic unit of action organization. CodeAct~\citep{wang2024executable} uses executable code to express control flow and orchestrate multi-tool interactions, reducing dependence on long natural-language chains, but it still requires generating task-specific code from scratch and does not directly support accumulation of reusable patterns across tasks. Voyager~\citep{wang2023voyager}, Ghost in the Minecraft~\citep{zhu2023ghost} demonstrate that agents can build up a library of reusable code skills through autonomous exploration and curriculum learning, but the portability of those skills is bounded by the closed rules and state spaces of game environments. CREATOR~\citep{qian2023creator} and related methods attempt to abstract reusable components from recurring patterns, yet they typically provide limited systematic evidence of cross-task generalization in realistic tool ecosystems. In more traditional planning paradigms, Hierarchical Task Networks (HTN)~\citep{georgievski2014overview} emphasize recursive decomposition of complex problems into reusable sub-skills that support compositional generalization. While effective in robotics, these approaches become challenging when adapted to LLM agents because debugging, credit assignment, and error propagation can quickly dominate as hierarchies deepen. Anthropic Skills~\citep{anthropic_agentskills} provides an engineering-oriented mechanism for packaging reusable workflows as explicit skill modules, but these modules are primarily authored and configured by humans rather than autonomously discovered by an agent at test time. SkillCraft targets this gap by providing a minimal MCP protocol with save, execute, list, and get primitives that lets agents automatically abstract successful tool sequences into verified, executable skills. By evaluating cross-task reuse, cross-difficulty transfer, and cross-model transfer, SkillCraft turns efficiency gains and skill generalization into quantifiable dimensions that can be compared across agent designs and tool-use settings.


Tool-use benchmarks mainly differ in the realism of tool executability and in whether tasks require long-horizon composition. In controlled settings, BFCL~\citep{patilberkeley} reduces tool use to structured function-parameter prediction, while $\tau$-Bench and ACEBench emphasize multi-turn interaction and correct tool selection under reproducible environments~\citep{yao2024tau, chen2025acebench}. Gorilla and AgentBench broaden tool and domain coverage~\citep{patil2023gorilla, liu2023agentbench}, but primarily evaluate API selection, such that short tool-call chains often suffice.

% ToolLLM~\citep{qin2023toolllm} studies tool retrieval and fine-tuning, yet its signal is still dominated by single-use accuracy rather than reuse of recurring patterns. 
More realistic benchmarks execute tools in richer environments. AppWorld supports application-level state transitions~\citep{trivedi2024appworld}, and MCP-based suites such as MCPWorld, MCP-RADAR, MCPEval, MCP-AgentBench, LiveMCPBench, and MCPAtlas standardize tool integration across servers~\citep{yan2025mcpworld, gao2025mcp, liu2025mcpeval, guo2025mcp, mo2025livemcpbench, bandimcp}, though tasks often remain single-application with simplified initial states. WebArena, OSWorld, SWE-Bench, and TheAgentCompany emphasize long-horizon execution and error recovery in web, desktop, and code workflows~\citep{zhou2023webarena, xie2024osworld, jimenez2024swebench, xu2024theagentcompanybenchmarkingllmagents}, while GAIA, ARE, and BrowseComp focus on everyday tasks and web-based information seeking~\citep{mialon2023gaia, froger2025scaling, wei2025browsecomp}. Tool Decathlon (Toolathlon) further consolidates real tools, fuzzy instructions, execution verification, and cross-application workflows~\citep{li2025tool}.
% SkillCraft complements this line by structuring tasks as families with recurring structure and scalable difficulty to make procedure reuse measurable.

On the pipeline side, most tool-using agents follow the “reasoning–acting–observing" loop introduced by ReAct~\citep{yao2023react}, where planning and state tracking are repeated at every tool call. CodeAct~\citep{wang2024executable} shifts the action space to executable code to express control flow and multi-tool orchestration, but it still regenerates code per task and does not accumulate reusable procedures. Voyager~\citep{wang2023voyager} and Ghost in the Minecraft~\citep{zhu2023ghost} show that agents can grow a code skill library through exploration, yet the resulting skills are tied to game rules and state spaces. CREATOR~\citep{qian2023creator} abstracts reusable components from patterns but provides limited evidence of robust cross-task generalization in realistic tool ecosystems. 
% Hierarchical Task Networks~\citep{georgievski2014overview} formalize recursive decomposition into sub-skills, but when adapted to LLM agents they face cascading errors and high debugging overhead. 
Anthropic Skills~\citep{anthropic_agentskills} packages workflows as explicit skill modules, but these modules are typically authored and configured by humans rather than induced from execution. In contrast, SkillCraft enables autonomous reuse with a minimal MCP protocol that compiles successful tool sequences into verified executable skills.

% =========================================================
\section{Skill Mode: System Details}
\label{app:system}

\subsection{Four primitive tools enabling Skill Mode}
\label{app:primitive_tools}
We illustrate the detailed design and functionality of the four primitive tools that together enable the proposed Skill Mode in Figure~\ref{fig:macro-tools}.
\begin{figure*}[t!]
\centering
\setlength{\tabcolsep}{0pt}
\begin{tabular}{@{}p{0.495\textwidth}@{\hspace{0.01\textwidth}}p{0.495\textwidth}@{}}

% Tool 1
\begin{toolbox}[equal height group=row1]{Tool 1: save\_macro}\footnotesize\color{black}
\textbf{Description:} ~\textcolor{red}{Save a reusable workflow} as executable macro.

\vspace{2mm}
\textbf{Input:}\\[1mm]
\small
\texttt{macro\_name} -- Unique identifier\\
\texttt{script\_code} -- Python script to execute\\
\texttt{parameters} -- List of variable names\\
\texttt{description} -- Human-readable summary

\vspace{2mm}
\textbf{Output:}\\[1mm]
\small\texttt{"Macro '<name>' saved successfully."}

\vspace{2mm}
\textbf{When to Use:}\\[1mm]
\small When identifying ~\textcolor{red}{repetitive workflows}. External tools can be called inside the script.
\end{toolbox}
&
% Tool 2
\begin{toolbox}[equal height group=row1]{Tool 2: execute\_macro}\footnotesize\color{black}
\textbf{Description:} Run a saved macro with ~\textcolor{red}{new arguments}.

\vspace{2mm}
\textbf{Input:}\\[1mm]
\small
\texttt{macro\_name} -- Name of saved macro\\
\texttt{args} -- Dict of arguments

\vspace{2mm}
\textbf{Output:}\\[1mm]
\small
\texttt{\{"status": "success", "result": <data>\}}\\
\texttt{\{"status": "failed", "result": <error>\}}

\vspace{2mm}
\textbf{When to Use:}\\[1mm]
\small ~\textcolor{red}{For batch tasks with repetitive logic.} Replaces sequential calls with macro executions.
\end{toolbox}
\\[4mm]

% Tool 3
\begin{toolbox}[equal height group=row2]{Tool 3: list\_macros}\footnotesize\color{black}
\textbf{Description:} List ~\textcolor{red}{all macros} saved in current session.

\vspace{2mm}
\textbf{Input:}\\[1mm]
\small\textit{(No parameters required)}

\vspace{2mm}
\textbf{Output:}\\[1mm]
\small
\texttt{Macro 1: <name> -- <description>}\\
\texttt{Macro 2: <name> -- <description>}\\
\texttt{...}

\vspace{2mm}
\textbf{When to Use:}\\[1mm]
\small Check available macros ~\textcolor{red}{before} creating new or reusing existing ones.
\end{toolbox}
&
% Tool 4
\begin{toolbox}[equal height group=row2]{Tool 4: get\_macro}\footnotesize\color{black}
\textbf{Description:} ~\textcolor{red}{Retrieve full source code} of a saved macro.

\vspace{2mm}
\textbf{Input:}\\[1mm]
\small\texttt{macro\_name} -- Name of target macro

\vspace{2mm}
\textbf{Output:}\\[1mm]
\small
\texttt{\{"script\_code": "...",}\\
\texttt{~~"parameters": ["path"],}\\
\texttt{~~"version": 1\}}

\vspace{2mm}
\textbf{When to Use:}\\[1mm]
\small ~\textcolor{red}{Inspect or debug a macro} before executing it on new data.
\end{toolbox}
\\
\end{tabular}
\caption{Pseudo-code for the four primitive tools that enable Skill Mode.}
\label{fig:macro-tools}
\end{figure*}



\subsection{Why Skill Mode improves efficiency}
\label{app:why_efficiency}
Figure~\ref{fig:why-figure} illustrates why Skill Mode improves efficiency through two complementary mechanisms. In normal tool use, raw tool outputs (e.g., full webpages or verbose API responses) are repeatedly injected into the context, bloating the prompt with extraneous information and incurring repeated argument-passing costs as the output of one tool becomes the input of the next via the agent. Skill Mode instead extracts and caches only the minimal, task-relevant fields, enabling direct tool-to-tool chaining and allowing intermediate results to be passed once rather than re-serialized at every step. Moreover, by reusing previously discovered tool sequences as atomic skills, the agent amortizes planning and reasoning cost over repeated executions, avoiding the need to reconstruct the same multi-step workflow from scratch.

\begin{figure*}[t!]
    \centering
    \includegraphics[width=\textwidth]{Figures/Fig5.pdf}
    \caption{Skill Mode improves efficiency through two mechanisms. First, it reduces argument passing overhead by enabling direct tool chaining (Tool A $\rightarrow$ Tool B $\rightarrow$ Tool C) rather than shuttling intermediate outputs through the agent (Tool A $\rightarrow$ Agent $\rightarrow$ Tool B $\rightarrow$ Agent $\rightarrow$ Tool C). Second, it amortizes planning cost by allowing agents to reuse previously discovered tool sequences, eliminating the need to reason about recurring multi-step patterns from scratch.}
    \label{fig:why-figure}
\end{figure*}

\subsection{Implementation details}
\label{app:implementation_details}
This section provides additional implementation details that complement the methodology described in the main text.

\paragraph{Execution Configuration.}
To ensure reproducibility and prevent resource exhaustion, we impose several execution limits on each task. Each task is allocated a maximum of 150 conversation turns (or 300 steps in single-turn mode) and a 60-minute timeout. We enforce cumulative token limits of 1M input tokens and 150K output tokens per task, with individual requests capped at 150K input tokens. Tasks exceeding these limits are terminated and evaluated based on partial completion. For generation, all models use temperature$=$0.0 and top\_p$=$1.0 to ensure deterministic outputs. We set \texttt{tool\_choice="auto"} to allow models to decide when to invoke tools autonomously.

\paragraph{Skill Storage and Execution.}
Skills are persisted as JSON entries in a \texttt{skill\_cache.json} file within each task's workspace. Each skill entry contains: (1) \texttt{script\_code}---executable Python code that invokes tools via a \texttt{call\_tool(name, **kwargs)} interface, (2) \texttt{parameters}---a list of input parameter names, (3) \texttt{description}---natural language documentation, and (4) \texttt{execution\_stats}---runtime statistics tracking successful and failed executions.

\paragraph{Evaluation Protocol.}
We employ a partial-credit scoring system where each task defines multiple weighted evaluation criteria. Typical criteria include: output file existence (10 points), JSON validity (10 points), data completeness (30 points), and field-level accuracy (50 points). A task is considered \emph{successful} if it achieves $\geq$90\% of the maximum score. Efficiency metrics (tokens, cost, turns, tool calls) are computed only over tasks where \emph{both} baseline and skill modes succeed, ensuring fair comparison. All API costs are tracked via the OpenRouter billing API.


% =========================================================
\section{SkillCraft: Benchmark Construction Details}
\label{app:benchmark}

\subsection{Task API Sources}
\label{app:task_sources}

\definecolor{urlblue}{RGB}{0, 180, 200}
\begin{table*}[h]
\centering
\small
\renewcommand{\arraystretch}{1.15}
\setlength{\tabcolsep}{5pt}

\caption{Complete list of API sources used in \textsc{SkillCraft}. The benchmark comprises 21 task families across 6 domains (Entertainment, Reference, Education, Developer, Science, Food). \textbf{Tools}: number of distinct API-wrapping functions per task. Each task family includes 6 difficulty-scaled variants (Easy: 3 subtasks, Medium: 5 subtasks, Hard: 7 subtasks), totaling 126 tasks. All APIs except Local DNA Analysis are publicly available REST endpoints.}

\begin{tabular}{llcl}
\toprule
\textbf{Task Family} & \textbf{Domain} & \textbf{Tools} & \textbf{Source} \\
\midrule
Cat Facts Collector & Reference & 5 & \textcolor{urlblue}{\url{https://catfact.ninja}} \\
Cocktail Menu Generator & Food & 5 & \textcolor{urlblue}{\url{https://thecocktaildb.com}} \\
Countries Encyclopedia & Reference & 5 & \textcolor{urlblue}{\url{https://restcountries.com}} \\
D\&D Campaign Builder & Gaming & 6 & \textcolor{urlblue}{\url{https://dnd5eapi.co}} \\
D\&D Monster Compendium & Gaming & 6 & \textcolor{urlblue}{\url{https://dnd5eapi.co}} \\
Dog Breeds Encyclopedia & Reference & 5 & \textcolor{urlblue}{\url{https://dog.ceo/api}} \\
GitLab Deep Analysis & Developer & 6 & \textcolor{urlblue}{\url{https://gitlab.com/api/v4}} \\
Jikan Anime Analysis & Entertainment & 5 & \textcolor{urlblue}{\url{https://api.jikan.moe}} \\
JSONPlaceholder Analyzer & Developer & 7 & \textcolor{urlblue}{\url{https://jsonplaceholder.typicode.com}} \\
Local DNA Analysis & Science & 5 & Custom Implementation \\
Name Demographics & Society & 5 & \textcolor{urlblue}{\url{https://genderize.io}} \\
Open-Meteo Weather & Science & 5 & \textcolor{urlblue}{\url{https://open-meteo.com}} \\
PokéAPI Pokédex & Gaming & 5 & \textcolor{urlblue}{\url{https://pokeapi.co}} \\
Random User Database & Society & 5 & \textcolor{urlblue}{\url{https://randomuser.me}} \\
Recipe Cookbook Builder & Food & 6 & \textcolor{urlblue}{\url{https://themealdb.com}} \\
Rick \& Morty Explorer & Entertainment & 5 & \textcolor{urlblue}{\url{https://rickandmortyapi.com}} \\
TVMaze Series Analyzer & Developer & 5 & \textcolor{urlblue}{\url{https://api.tvmaze.com}} \\
University Directory & Education & 5 & \textcolor{urlblue}{\url{http://universities.hipolabs.com}} \\
USGS Earthquake Monitor & Science & 6 & \textcolor{urlblue}{\url{https://earthquake.usgs.gov}} \\
Vocabulary Builder & Reference & 5 & \textcolor{urlblue}{\url{https://dictionaryapi.dev}} \\
World Bank Snapshot & Education & 5 & \textcolor{urlblue}{\url{https://api.worldbank.org}} \\
\bottomrule
\end{tabular}

\label{tab:task_sources}
\end{table*}

We present the complete list of API sources used in \textsc{SkillCraft} benchmark in Table~\ref{tab:task_sources}. 
Our 21 task families span six application domains—from entertainment and gaming to science and development—covering a diverse range of real-world API interaction patterns.
All APIs are publicly available REST endpoints that require structured multi-step interactions, making them ideal candidates for evaluating skill composition and reuse.
For each task family, we implement 5--7 tool functions wrapping distinct API endpoints; difficulty levels (Easy/Medium/Hard) control the number of subtasks (3/4/5) and thus total API calls required per task.
Most of these APIs are sourced from existing community-maintained projects, while the Local DNA Analysis task uses a custom implementation for bioinformatics operations.


% =========================================================
\section{Additional Analyses}
\label{app:analysis}



\subsection{Results by task difficulty}
\label{app:difficulty_analysis}
\begin{table*}[h]
\centering
\small
\renewcommand{\arraystretch}{1.15}
\setlength{\tabcolsep}{3pt}
\caption{
Results breakdown by difficulty level (Easy: e1--e3, Medium: m1--m2, Hard: h1).
\textbf{Success Rate}: task completion rate (score $\geq$ 90) for \colorbox{baseblue}{Baseline} and \colorbox{skillgreen}{Skill} modes.
\textbf{Skill Stats}: Exec = skill execution success rate; Reuse = average times each skill is invoked.
\textbf{Efficiency metrics}: per-task averages computed over tasks where \emph{both} modes succeeded.
\textbf{Diff}: percentage change; \better{negative} = improvement, \worse{positive} = degradation.
}

\resizebox{\textwidth}{!}{%
\begin{tabular}{l|l|cc|cc|ccc|ccc|ccc|ccc}
\toprule
\multirow{2}{*}{\textbf{Model}} &
\multirow{2}{*}{\textbf{Diff.}} &
\multicolumn{2}{c|}{\textbf{Success Rate}} &
\multicolumn{2}{c|}{\textbf{Skill Stats}} &
\multicolumn{3}{c|}{\textbf{Avg Tokens}} &
\multicolumn{3}{c|}{\textbf{Avg Cost (\$)}} &
\multicolumn{3}{c|}{\textbf{Avg Turns}} &
\multicolumn{3}{c}{\textbf{Avg Tool Calls}} \\
\cmidrule{3-4}\cmidrule{5-6}\cmidrule{7-9}\cmidrule{10-12}\cmidrule{13-15}\cmidrule{16-18}
& &
\cellcolor{baseblue}\textbf{Base} & \cellcolor{skillgreen}\textbf{Skill} &
\textbf{Exec} & \textbf{Reuse} &
\cellcolor{baseblue}\textbf{Base} & \cellcolor{skillgreen}\textbf{Skill} & \textbf{Diff} &
\cellcolor{baseblue}\textbf{Base} & \cellcolor{skillgreen}\textbf{Skill} & \textbf{Diff} &
\cellcolor{baseblue}\textbf{Base} & \cellcolor{skillgreen}\textbf{Skill} & \textbf{Diff} &
\cellcolor{baseblue}\textbf{Base} & \cellcolor{skillgreen}\textbf{Skill} & \textbf{Diff} \\
\midrule
\multirow{3}{*}{\textbf{Kimi-K2-Thinking}} & Easy & 30/63 (48\%) & 32/63 (51\%) & 81\% & 2.6$\times$ & 427K & 293K & \better{-31\%} & 0.17 & 0.12 & \better{-29\%} & 9.0 & 13.7 & \worse{+53\%} & 12.2 & 10.6 & \better{-13\%} \\
 & Medium & 17/42 (40\%) & 17/42 (40\%) & 76\% & 2.9$\times$ & 576K & 335K & \better{-42\%} & 0.23 & 0.14 & \better{-39\%} & 8.1 & 15.1 & \worse{+87\%} & 19.5 & 12.5 & \better{-36\%} \\
 & Hard & 8/21 (38\%) & 7/21 (33\%) & 66\% & 4.8$\times$ & 622K & 285K & \better{-54\%} & 0.25 & 0.13 & \better{-50\%} & 10.4 & 16.4 & \worse{+58\%} & 27.8 & 18.4 & \better{-34\%} \\
\midrule
\multirow{3}{*}{\textbf{DeepSeek-V3.2-EXP}} & Easy & 42/63 (66\%) & 47/63 (74\%) & 94\% & 2.5$\times$ & 943K & 512K & \better{-46\%} & 0.27 & 0.22 & \better{-18\%} & 28.2 & 25.7 & \better{-9\%} & 15.4 & 13.2 & \better{-14\%} \\
 & Medium & 25/42 (59\%) & 25/42 (59\%) & 89\% & 3.2$\times$ & 1.34M & 556K & \better{-59\%} & 0.22 & 0.08 & \better{-63\%} & 36.2 & 30.4 & \better{-16\%} & 23.4 & 17.2 & \better{-26\%} \\
 & Hard & 9/21 (42\%) & 15/21 (71\%) & 88\% & 4.4$\times$ & 844K & 547K & \better{-35\%} & 0.22 & 0.07 & \better{-69\%} & 21.4 & 33.1 & \worse{+55\%} & 28.7 & 17.6 & \better{-39\%} \\
\midrule
\multirow{3}{*}{\textbf{DeepSeek-R1}} & Easy & 50/63 (79\%) & 55/63 (87\%) & 68\% & 2.9$\times$ & 498K & 470K & \better{-6\%} & 0.21 & 0.20 & \better{-3\%} & 14.0 & 16.8 & \worse{+20\%} & 11.1 & 11.0 & \better{-1\%} \\
 & Medium & 28/42 (66\%) & 31/42 (74\%) & 77\% & 3.5$\times$ & 631K & 241K & \better{-62\%} & 0.26 & 0.13 & \better{-50\%} & 13.6 & 14.9 & \worse{+10\%} & 15.8 & 10.7 & \better{-32\%} \\
 & Hard & 11/21 (52\%) & 15/21 (71\%) & 68\% & 4.2$\times$ & 855K & 421K & \better{-51\%} & 0.34 & 0.17 & \better{-49\%} & 17.1 & 16.9 & \better{-1\%} & 16.7 & 16.8 & \worse{+1\%} \\
\midrule
\multirow{3}{*}{\textbf{GLM-4.7}} & Easy & 52/63 (82\%) & 57/63 (90\%) & 90\% & 2.9$\times$ & 661K & 428K & \better{-35\%} & 0.18 & 0.12 & \better{-33\%} & 12.2 & 11.4 & \better{-6\%} & 13.6 & 11.8 & \better{-13\%} \\
 & Medium & 27/42 (64\%) & 36/42 (85\%) & 89\% & 4.0$\times$ & 874K & 514K & \better{-41\%} & 0.22 & 0.10 & \better{-54\%} & 13.7 & 15.3 & \worse{+11\%} & 19.0 & 15.1 & \better{-21\%} \\
 & Hard & 12/21 (57\%) & 15/21 (71\%) & 94\% & 4.9$\times$ & 1.17M & 648K & \better{-45\%} & 0.28 & 0.16 & \better{-43\%} & 19.6 & 15.1 & \better{-23\%} & 28.2 & 16.5 & \better{-41\%} \\
\midrule
\multirow{3}{*}{\textbf{Gemini 3 Pro}} & Easy & 55/63 (87\%) & 59/63 (93\%) & 95\% & 2.3$\times$ & 534K & 300K & \better{-44\%} & 0.46 & 0.30 & \better{-35\%} & 15.4 & 19.6 & \worse{+27\%} & 12.8 & 9.2 & \better{-29\%} \\
 & Medium & 37/42 (88\%) & 40/42 (95\%) & 92\% & 2.7$\times$ & 730K & 323K & \better{-56\%} & 0.68 & 0.31 & \better{-55\%} & 16.6 & 18.4 & \worse{+11\%} & 19.5 & 10.6 & \better{-46\%} \\
 & Hard & 16/21 (76\%) & 17/19 (89\%) & 96\% & 3.3$\times$ & 970K & 227K & \better{-77\%} & 0.90 & 0.27 & \better{-70\%} & 22.2 & 20.3 & \better{-8\%} & 28.9 & 8.7 & \better{-70\%} \\
\midrule
\multirow{3}{*}{\textbf{Minimax-M2.1}} & Easy & 59/63 (94\%) & 58/63 (96\%) & 100\% & 3.0$\times$ & 379K & 363K & \better{-4\%} & 0.04 & 0.03 & \better{-13\%} & 7.7 & 7.4 & \better{-4\%} & 12.1 & 11.4 & \better{-5\%} \\
 & Medium & 40/42 (95\%) & 41/42 (98\%) & 84\% & 3.6$\times$ & 468K & 380K & \better{-19\%} & 0.05 & 0.05 & \better{-8\%} & 7.9 & 7.1 & \better{-11\%} & 18.6 & 16.8 & \better{-10\%} \\
 & Hard & 18/21 (86\%) & 20/21 (95\%) & 100\% & 3.0$\times$ & 479K & 409K & \better{-15\%} & 0.06 & 0.05 & \better{-4\%} & 8.2 & 7.2 & \better{-12\%} & 26.8 & 24.8 & \better{-8\%} \\
\midrule
\multirow{3}{*}{\textbf{Claude 4.5 Sonnet}} & Easy & 60/63 (95\%) & 60/63 (95\%) & 99\% & 3.0$\times$ & 1.06M & 399K & \better{-62\%} & 0.81 & 0.25 & \better{-69\%} & 19.2 & 17.4 & \better{-9\%} & 11.4 & 9.0 & \better{-21\%} \\
 & Medium & 39/42 (92\%) & 41/42 (98\%) & 100\% & 3.7$\times$ & 1.54M & 369K & \better{-76\%} & 1.32 & 0.27 & \better{-80\%} & 22.0 & 17.7 & \better{-19\%} & 15.7 & 9.0 & \better{-43\%} \\
 & Hard & 20/21 (95\%) & 20/21 (95\%) & 98\% & 4.7$\times$ & 1.96M & 440K & \better{-77\%} & 1.46 & 0.40 & \better{-72\%} & 25.5 & 19.9 & \better{-22\%} & 20.3 & 10.2 & \better{-50\%} \\
\midrule
\multirow{3}{*}{\textbf{GPT-5.2}} & Easy & 59/63 (94\%) & 60/63 (95\%) & 91\% & 3.0$\times$ & 939K & 196K & \better{-79\%} & 1.38 & 0.30 & \better{-78\%} & 22.3 & 15.1 & \better{-32\%} & 12.5 & 7.7 & \better{-38\%} \\
 & Medium & 34/42 (81\%) & 37/42 (88\%) & 95\% & 4.2$\times$ & 1.44M & 314K & \better{-78\%} & 2.10 & 0.48 & \better{-77\%} & 26.9 & 17.0 & \better{-37\%} & 21.0 & 8.9 & \better{-58\%} \\
 & Hard & 16/21 (76\%) & 17/21 (80\%) & 90\% & 4.3$\times$ & 1.86M & 405K & \better{-78\%} & 2.72 & 0.61 & \better{-77\%} & 31.4 & 20.5 & \better{-35\%} & 41.3 & 13.4 & \better{-68\%} \\
\bottomrule
\end{tabular}
}

\label{tab:difficulty_breakdown}
\end{table*}

% Keep your existing text exactly as-is below:
Table~\ref{tab:difficulty_breakdown} presents a detailed breakdown of our experimental results across three difficulty levels: Easy (tasks e1--e3), Medium (tasks m1--m2), and Hard (task h1). We identify several noteworthy patterns that provide deeper insights into the behavior and benefits of skill reuse.

\paragraph{Skill Reuse Frequency Increases with Task Complexity.}
Across all models, the average skill reuse count shows a consistent upward trend with task difficulty. For Easy tasks, skills are invoked 2.3--3.0$\times$ on average, while Hard tasks see 3.0--4.9$\times$ reuse. This pattern reflects the compositional nature of our benchmark: harder tasks require more repeated API compositions, which naturally leads to more opportunities for skill reuse. Notably, GLM-4.7 achieves the highest reuse rate (4.9$\times$) on Hard tasks, demonstrating effective skill generalization across complex scenarios.

\paragraph{Efficiency Gains are More Pronounced on Harder Tasks.}
Token savings exhibit a clear correlation with task difficulty. For frontier models like Claude 4.5 Sonnet and GPT-5.2, token reduction on Hard tasks reaches 77--78\%, compared to 62--79\% on Easy tasks. Similarly, tool call reduction is most dramatic on Hard tasks: Gemini 3 Pro achieves a 70\% reduction on Hard versus 29\% on Easy, while GPT-5.2 shows 68\% versus 38\%. This suggests that skill reuse provides greater benefits when tasks involve more complex, multi-step API orchestrations---precisely the scenarios where manual tool composition becomes most costly.

\paragraph{Success Rate Improvements Favor Challenging Tasks.}
For models with moderate baseline performance, skill reuse disproportionately improves success rates on Hard tasks. DeepSeek-V3.2-EXP shows a remarkable +29 percentage point improvement on Hard tasks (from 42\% to 71\%) compared to only +8 points on Easy tasks. Similarly, DeepSeek-R1 improves by +19 points on Hard versus +7 points on Easy. This indicates that skills learned from easier variants effectively transfer to help models overcome challenges they would otherwise fail, validating the cross-difficulty generalization capability of our skill framework.

\paragraph{High-Capacity Models Benefit from Efficiency, Not Accuracy.}
Frontier models (Claude, GPT-5.2) already achieve $>$95\% success rates on Easy tasks in baseline mode, leaving little room for accuracy improvement. However, they show the largest efficiency gains: Claude achieves 72\% average token reduction, and GPT-5.2 achieves 78\%. In contrast, Minimax-M2.1, which exhibits highly efficient baseline behavior (only 379K--479K tokens per task), shows modest 4--19\% token savings. This suggests that skill reuse is most valuable for models whose baseline execution involves verbose, sequential API interactions.

\paragraph{Skill Execution Remains Robust Across Difficulties.}
Skill execution success rates remain consistently high (66--100\%) across all difficulty levels for most models, indicating that skills created during easier tasks transfer reliably to harder contexts. The lowest execution rates appear in Kimi-K2-Thinking (66\% on Hard) and DeepSeek-R1 (68\% on Easy/Hard), both of which employ extended reasoning that may occasionally conflict with deterministic skill execution patterns.

\subsection{Direct execution mode}
\label{app:direct_exec}
% Move the table close to where you discuss it:
% \newcommand{\better}[1]{\textcolor{green!60!black}{#1}}
% \newcommand{\worse}[1]{\textcolor{red!70!black}{#1}}
\definecolor{baseblue}{RGB}{230, 240, 250}
\definecolor{skillgreen}{RGB}{230, 245, 235}
\definecolor{directexecpurple}{RGB}{240, 230, 250}

\begin{table*}[!h]
\centering
\small
\renewcommand{\arraystretch}{1.15}
\setlength{\tabcolsep}{3.5pt}
\caption{Three-mode comparison (Base, Skill, Direct Exec) on 48-task subset. \textbf{Base}: No skill library. \textbf{Skill}: With skill library from previous runs. \textbf{Direct Exec}: Skills are directly executed without agent intervention. Efficiency metrics are computed over tasks where both Base and the respective mode succeeded.}
\resizebox{\textwidth}{!}{%
\begin{tabular}{l|l|cc|cc|cc|cc|cc|cc}
\toprule
\multirow{2}{*}{\textbf{Model}} & \multirow{2}{*}{\textbf{Mode}} & \multicolumn{2}{c|}{\textbf{Success Rate}} & \multicolumn{2}{c|}{\textbf{Skill Stats}} & \multicolumn{2}{c|}{\textbf{Avg Tokens}} & \multicolumn{2}{c|}{\textbf{Avg Cost (\$)}} & \multicolumn{2}{c|}{\textbf{Avg Turns}} & \multicolumn{2}{c}{\textbf{Avg Tool Calls}} \\
\cmidrule{3-4}\cmidrule{5-6}\cmidrule{7-8}\cmidrule{9-10}\cmidrule{11-12}\cmidrule{13-14}
 &  & \textbf{Succ} & \textbf{Rate} & \textbf{Exec} & \textbf{Reuse} & \textbf{Val} & \textbf{Diff} & \textbf{Val} & \textbf{Diff} & \textbf{Val} & \textbf{Diff} & \textbf{Val} & \textbf{Diff} \\
\midrule
\textbf{Claude-4.5-Sonnet} & \cellcolor{baseblue}Base & 47/48 & 98\% & -- & -- & 1.72M & -- & 1.73 & -- & 15.7 & -- & 14.7 & -- \\
 & \cellcolor{skillgreen}Skill & 43/48 & 90\% & 99\% & 3.7$\times$ & 0.34M & \better{-80\%} & 0.22 & \better{-87\%} & 10.5 & \better{-33\%} & 9.5 & \better{-36\%} \\
 & \cellcolor{directexecpurple}Direct Exec & 46/48 & 96\% & 68\% & 3.1$\times$ & 0.16M & \better{-90\%} & 0.17 & \better{-89\%} & 5.8 & \better{-64\%} & 4.8 & \better{-68\%} \\
\midrule
\textbf{GPT-5.2} & \cellcolor{baseblue}Base & 45/48 & 94\% & -- & -- & 1.18M & -- & 1.52 & -- & 24.5 & -- & 23.1 & -- \\
 & \cellcolor{skillgreen}Skill & 43/48 & 90\% & 97\% & 3.5$\times$ & 0.26M & \better{-78\%} & 0.39 & \better{-74\%} & 8.9 & \better{-64\%} & 7.9 & \better{-66\%} \\
 & \cellcolor{directexecpurple}Direct Exec & 41/48 & 85\% & 68\% & 3.1$\times$ & 0.06M & \better{-95\%} & 0.14 & \better{-91\%} & 4.5 & \better{-78\%} & 3.5 & \better{-81\%} \\
\bottomrule
\end{tabular}
}
\label{tab:direct_exec}
\end{table*}

We further investigate the efficiency impact of script parameterization by implementing \textbf{Direct Exec Mode}, an alternative approach that trades generalization capability for execution efficiency.

In our Skill mode, agents create parameterized skills through a two-step process: first \texttt{save\_skill} to store a reusable script with parameter placeholders, then \texttt{execute\_skill} to invoke it with specific arguments. This design enables skill reuse across similar tasks but introduces overhead from parameter abstraction and the save-then-execute workflow.

Direct Exec Mode takes a fundamentally different approach. Instead of creating generalizable skills, agents write \textbf{single-use scripts} with all values \textbf{hardcoded directly} into the code. The agent uses \texttt{exec\_script} to execute these scripts immediately, after which they are discarded. This eliminates both the abstraction overhead of designing reusable interfaces and the two-step save-execute workflow.

Table~\ref{tab:direct_exec} compares Base, Skill, and Direct Exec on a 48-task subset. For Claude-4.5-Sonnet, Direct Exec largely preserves success at 96\% while cutting tokens from 1.72M to 0.16M, and it reduces turns from 15.7 to 5.8 with tool calls from 14.7 to 4.8. Skill mode is less aggressive at 0.34M tokens and it drops success to 90\%. For GPT-5.2, Direct Exec achieves the largest savings from 1.18M to 0.06M tokens and reduces turns from 24.5 to 4.5, but success falls from 94\% to 85\%, while Skill keeps 90\% at 0.26M tokens. Direct Exec also has lower Exec at 68\% versus 97\% to 99\% in Skill mode, matching the fact that removing the agent loop removes recovery and adaptation. These results show Direct Exec as the efficiency upper bound when Skills transfer cleanly as standalone programs.
This advantage stems from two factors: (1) \textbf{reduced cognitive load}---the agent need not design generalizable parameter interfaces or anticipate future reuse scenarios; and (2) \textbf{simplified execution}---hardcoded values eliminate potential parameter binding errors that can occur in parameterized skill execution.

These results suggest that the generalization capability of Skills incurs a non-trivial overhead. When tasks are isolated and patterns are unlikely to be reused, Direct Exec Mode provides a more efficient alternative.


\subsection{Trajectory analysis}
\label{app:traj}
% =========================================================================
% Trajectory Display - Clean Deep Purple Theme
% =========================================================================

% Preamble (add to your document preamble)
% \usepackage{tcolorbox}
% \usepackage{xcolor}
% \tcbuselibrary{skins,breakable}

% Define colors - Deep Purple Theme Only
\definecolor{TrajPurple}{HTML}{4A148C}      % Deep purple - main color
\definecolor{TrajPurpleLight}{HTML}{7B1FA2} % Lighter purple for accents
\definecolor{TrajPurpleBg}{HTML}{F3E5F5}    % Very light purple background
\definecolor{TrajGray}{HTML}{424242}        % Dark gray for text
\definecolor{TrajLightGray}{HTML}{F5F5F5}   % Light gray for backgrounds

% Counter for steps
\newcounter{trajstep}
\newcommand{\TrajReset}{\setcounter{trajstep}{0}}

% Main trajectory environment
\newtcolorbox{trajectory}[1]{
  enhanced,
  breakable,
  colback=white,
  colframe=TrajPurple,
  boxrule=1pt,
  arc=3mm,
  left=4mm,
  right=4mm,
  top=3mm,
  bottom=3mm,
  fonttitle=\bfseries\normalsize,
  coltitle=white,
  colbacktitle=TrajPurple,
  title=#1,
  attach boxed title to top left={yshift=-2mm, xshift=-1mm},
  boxed title style={boxrule=0pt, arc=2mm, left=3mm, right=3mm, xshift=0mm}
}

% System/User card - light gray background
\newtcolorbox{syscard}[1]{
  enhanced,
  colback=TrajLightGray,
  colframe=TrajLightGray,
  boxrule=0pt,
  arc=2mm,
  left=3mm,
  right=3mm,
  top=2mm,
  bottom=2mm,
  before skip=2mm,
  after skip=2mm,
  fontupper=\small,
  before upper={\textbf{#1}\par\smallskip}
}

% Agent step card - purple left border
\newtcolorbox{stepcard}{
  enhanced,
  colback=TrajPurpleBg,
  colframe=TrajPurpleBg,
  boxrule=0pt,
  arc=2mm,
  left=3mm,
  right=3mm,
  top=2mm,
  bottom=2mm,
  before skip=2mm,
  after skip=2mm,
  fontupper=\small,
  borderline west={2pt}{0pt}{TrajPurple}
}

% Response card - no border, just light background
\newtcolorbox{respcard}{
  enhanced,
  colback=TrajLightGray,
  colframe=TrajLightGray,
  boxrule=0pt,
  arc=2mm,
  left=3mm,
  right=3mm,
  top=2mm,
  bottom=2mm,
  before skip=1mm,
  after skip=2mm,
  fontupper=\small
}

% Statistics box
\newtcolorbox{statsbox}[1]{
  enhanced,
  colback=TrajPurpleBg,
  colframe=TrajPurple,
  boxrule=0.8pt,
  arc=2mm,
  left=3mm,
  right=3mm,
  top=2mm,
  bottom=2mm,
  before skip=3mm,
  fontupper=\small,
  title={\small\bfseries #1},
  coltitle=TrajPurple,
  colbacktitle=white,
  attach boxed title to top left={yshift=-2mm, xshift=3mm},
  boxed title style={boxrule=0.8pt, colframe=TrajPurple, arc=1.5mm, left=2mm, right=2mm}
}

% Inline code style - use italic for parameters
\newcommand{\tcode}[1]{\textit{#1}}

% Step command
\newcommand{\stepnum}{\refstepcounter{trajstep}\textbf{Step \#\thetrajstep}}

% Abbreviated steps
\newcommand{\abbrev}[1]{{\color{TrajGray}\itshape\small\hspace{3mm}#1}}

% =========================================================================
% CASE 1: Easy Task - Skill Creation Decision
% =========================================================================

\TrajReset
\begin{trajectory}{Traj A: Claude-4.5-Sonnet \quad\textit{\small cat-facts-collector/e2}}

\begin{syscard}{System}
You are an AI assistant with access to tools. Use save\_skill to create reusable workflows and execute\_skill to run them.
\end{syscard}

\begin{syscard}{User}
Collect information for 3 cat breeds (Persian, Siamese, Maine Coon) using breed\_profile, breed\_facts, and breed\_encyclopedia APIs. Save results to a JSON file.
\end{syscard}

\begin{stepcard}
\stepnum\quad Agent\par\smallskip
I'll process each breed systematically using direct API calls.\par
\textbf{call} \tcode{catfacts\_breed\_profile} \quad \tcode{Persian}
\end{stepcard}

\begin{respcard}
\{"origin": "Persia", "temperament": "Gentle", "life\_span": "12-17"...\}
\end{respcard}

\begin{stepcard}
\stepnum\quad Agent\par\smallskip
\textbf{call} \tcode{catfacts\_breed\_facts} \quad \tcode{Persian}
\end{stepcard}

\begin{respcard}
\{"fact\_1": "Persians are known for their long fur"...\}
\end{respcard}

\begin{stepcard}
\stepnum\quad Agent\par\smallskip
\textbf{call} \tcode{catfacts\_breed\_encyclopedia} \quad \tcode{Persian}
\end{stepcard}

\begin{respcard}
\{"description": "The Persian cat is a long-haired breed..."...\}
\end{respcard}

\begin{stepcard}
\stepnum\quad Agent\par\smallskip
Persian complete. Continuing with Siamese...
\end{stepcard}

\abbrev{Steps \#5--7: Same 3 API calls for Siamese and Maine Coon}

\begin{stepcard}
\textbf{Step \#8}\quad Agent\par\smallskip
\textbf{call} \tcode{write\_file} \quad \tcode{cat\_breeds\_report.json}
\end{stepcard}

\begin{respcard}
File written successfully.
\end{respcard}

\begin{stepcard}
\textbf{Step \#9}\quad Agent\par\smallskip
\textbf{call} \tcode{claim\_done} \quad Task completed successfully.
\end{stepcard}

\begin{statsbox}{Statistics --- Pass}
\textbf{Total Steps:} 9 \quad\quad \textbf{Total Tokens:} 762,052\par\smallskip
\hrule\smallskip
$\checkmark$ No skill created --- Direct API calls sufficient.
\end{statsbox}

\end{trajectory}

\vspace{4mm}

% =========================================================================
% Traj B
% =========================================================================

\TrajReset
\begin{trajectory}{Traj B: DeepSeek-V3.2 \quad\textit{\small cat-facts-collector/e2}}

\begin{syscard}{System}
You are an AI assistant with access to tools. Use save\_skill to create reusable workflows and execute\_skill to run them.
\end{syscard}

\begin{syscard}{User}
Collect information for 3 cat breeds (Persian, Siamese, Maine Coon) using breed\_profile, breed\_facts, and breed\_encyclopedia APIs. Save results to a JSON file.
\end{syscard}

\begin{stepcard}
\stepnum\quad Agent\par\smallskip
\textbf{call} \tcode{list\_directory} \quad \tcode{workspace}
\end{stepcard}

\begin{respcard}
[workspace contents]
\end{respcard}

\begin{stepcard}
\stepnum\quad Agent\par\smallskip
\textbf{call} \tcode{save\_skill} \quad \tcode{process\_cat\_breed}
\end{stepcard}

\begin{respcard}
Skill saved successfully.
\end{respcard}

\begin{stepcard}
\stepnum\quad Agent\par\smallskip
\textbf{call} \tcode{execute\_skill}\par
skill\_name: \tcode{process\_cat\_breed} \quad param: \tcode{Persian}
\end{stepcard}

\begin{respcard}
Skill executed.\par
$\triangleright$ Warning: missing breed\_facts, breed\_encyclopedia fields
\end{respcard}

\begin{stepcard}
\stepnum\quad Agent\par\smallskip
\textbf{call} \tcode{execute\_skill}\par
skill\_name: \tcode{process\_cat\_breed} \quad param: \tcode{Siamese}
\end{stepcard}

\begin{respcard}
Skill executed.\par
$\triangleright$ Warning: missing breed\_facts, breed\_encyclopedia fields
\end{respcard}

\begin{stepcard}
\stepnum\quad Agent\par\smallskip
\textbf{call} \tcode{execute\_skill}\par
skill\_name: \tcode{process\_cat\_breed} \quad param: \tcode{Maine Coon}
\end{stepcard}

\begin{respcard}
Skill executed.\par
$\triangleright$ Warning: missing breed\_facts, breed\_encyclopedia fields
\end{respcard}

\begin{stepcard}
\stepnum\quad Agent\par\smallskip
\textbf{call} \tcode{write\_file} \quad \tcode{partial\_results.json}
\end{stepcard}

\begin{respcard}
File written.
\end{respcard}

\begin{stepcard}
\stepnum\quad Agent\par\smallskip
\textbf{call} \tcode{file\_write\_json\_chunk} \quad \tcode{persian\_data}
\end{stepcard}

\begin{respcard}
Chunk written.
\end{respcard}

\abbrev{Steps \#8--14: More repair operations (write\_json\_chunk $\times$4, read\_file, file\_append)}

\begin{stepcard}
\textbf{Step \#15}\quad Agent\par\smallskip
\textbf{call} \tcode{claim\_done} \quad Task completed.
\end{stepcard}

\begin{statsbox}{Statistics --- Pass}
\textbf{Total Steps:} 15 \quad\quad \textbf{Total Tokens:} 1,507,830\par\smallskip
\hrule\smallskip
$\times$ Skill created unnecessarily --- 8 extra repair calls needed.
\end{statsbox}

\end{trajectory}

\vspace{4mm}

% =========================================================================
% CASE 2: Hard Task - Skill Quality Difference
% =========================================================================

\TrajReset
\begin{trajectory}{Traj C: Claude-4.5-Sonnet \quad\textit{\small cocktail-menu-generator/h1}}

\begin{syscard}{System}
You are an AI assistant with access to tools. Use save\_skill to create reusable workflows and execute\_skill to run them.
\end{syscard}

\begin{syscard}{User}
Create a cocktail menu for 5 cocktails (Margarita, Mojito, Old Fashioned, Martini, Negroni) using 5 APIs each. Save to JSON.
\end{syscard}

\begin{stepcard}
\stepnum\quad Agent\par\smallskip
25 calls total. I'll create a reusable skill.\par
\textbf{call} \tcode{save\_skill} \quad \tcode{process\_cocktail\_complete}
\end{stepcard}

\begin{respcard}
Skill saved successfully.
\end{respcard}

\begin{stepcard}
\stepnum\quad Agent\par\smallskip
Now processing all 5 cocktails...\par
\textbf{call} \tcode{execute\_skill}\par
skill\_name: \tcode{process\_cocktail\_complete} \quad param: \tcode{Margarita}
\end{stepcard}

\begin{respcard}
\{search, details, ingredient, category, glass\} $\rightarrow$ all fields complete
\end{respcard}

\abbrev{Steps \#3--6: execute\_skill for Mojito, Old Fashioned, Martini, Negroni $\rightarrow$ all success}

\begin{stepcard}
\textbf{Step \#7}\quad Agent\par\smallskip
\textbf{call} \tcode{write\_file} \quad \tcode{cocktail\_menu.json}
\end{stepcard}

\begin{respcard}
File written successfully.
\end{respcard}

\begin{stepcard}
\textbf{Step \#8}\quad Agent\par\smallskip
\textbf{call} \tcode{claim\_done} \quad Cocktail menu saved. Task completed.
\end{stepcard}

\begin{statsbox}{Statistics --- Pass}
\textbf{Total Steps:} 8 \quad\quad \textbf{Total Tokens:} 213,222\par\smallskip
\hrule\smallskip
$\checkmark$ 1 save $\rightarrow$ 5 executes --- Skill correct on first try.
\end{statsbox}

\end{trajectory}

\vspace{4mm}

% =========================================================================
% Traj D
% =========================================================================

\TrajReset
\begin{trajectory}{Traj D: DeepSeek-V3.2 \quad\textit{\small cocktail-menu-generator/h1}}

\begin{syscard}{System}
You are an AI assistant with access to tools. Use save\_skill to create reusable workflows and execute\_skill to run them.
\end{syscard}

\begin{syscard}{User}
Create a cocktail menu for 5 cocktails (Margarita, Mojito, Old Fashioned, Martini, Negroni) using 5 APIs each. Save to JSON.
\end{syscard}

\begin{stepcard}
\stepnum\quad Agent\par\smallskip
\textbf{call} \tcode{save\_skill} \quad \tcode{process\_cocktail}
\end{stepcard}

\begin{respcard}
Skill save failed.\par
$\triangleright$ SyntaxError: unexpected token `\}' at line 8
\end{respcard}

\begin{stepcard}
\stepnum\quad Agent\par\smallskip
Fix syntax error and retry...\par
\textbf{call} \tcode{save\_skill} \quad \tcode{process\_cocktail\_v2}
\end{stepcard}

\begin{respcard}
Skill saved.
\end{respcard}

\begin{stepcard}
\stepnum\quad Agent\par\smallskip
\textbf{call} \tcode{execute\_skill}\par
skill\_name: \tcode{process\_cocktail\_v2} \quad param: \tcode{Margarita}
\end{stepcard}

\begin{respcard}
Execution failed.\par
$\triangleright$ SyntaxError: `return' is invalid outside function (line 12)
\end{respcard}

\begin{stepcard}
\stepnum\quad Agent\par\smallskip
Skill failed. Fallback to direct API...\par
\textbf{call} \tcode{cocktail\_search} \quad \tcode{Margarita}
\end{stepcard}

\begin{respcard}
\{"id": "11007", "name": "Margarita"...\}
\end{respcard}

\begin{stepcard}
\stepnum\quad Agent\par\smallskip
\textbf{call} \tcode{cocktail\_details} \quad \tcode{11007}
\end{stepcard}

\begin{respcard}
\{"ingredients": ["Tequila", "Lime"...]...\}
\end{respcard}

\begin{stepcard}
\stepnum\quad Agent\par\smallskip
\textbf{call} \tcode{cocktail\_by\_ingredient} \quad \tcode{Tequila}
\end{stepcard}

\begin{respcard}
\{"cocktails": [...]...\}
\end{respcard}

\begin{stepcard}
\stepnum\quad Agent\par\smallskip
\textbf{call} \tcode{cocktail\_by\_ingredient} \quad \tcode{Lime}
\end{stepcard}

\begin{respcard}
\{"cocktails": [...]...\}
\end{respcard}

\begin{stepcard}
\stepnum\quad Agent\par\smallskip
\textbf{call} \tcode{save\_skill} \quad \tcode{process\_cocktail\_v3}
\end{stepcard}

\begin{respcard}
Skill saved.
\end{respcard}

\abbrev{Steps \#9--18: More execute\_skill calls + manual cleanup}

\begin{stepcard}
\textbf{Step \#19}\quad Agent\par\smallskip
\textbf{call} \tcode{claim\_done} \quad Task completed after multiple retries.
\end{stepcard}

\begin{statsbox}{Statistics --- Fail}
\textbf{Total Steps:} 19 \quad\quad \textbf{Total Tokens:} 1,141,166\par\smallskip
\hrule\smallskip
$\times$ 3 saves + fallback --- Skill quality poor, task failed.
\end{statsbox}

\end{trajectory}

% Trajectory Analysis Section
% To be included in the paper with: \input{Figures/trajectory_analysis}

We present representative trajectories from our experiments to illustrate the qualitative differences in how models approach skill creation and reuse. The above shows four trajectories: two from an easy task (\texttt{cat-facts-collector/e2}) and two from a hard task (\texttt{cocktail-menu-generator/h1}), comparing Claude-4.5-Sonnet and DeepSeek-V3.2.

\paragraph{Behavioral Divergence.}
A fundamental distinction emerges in how models decide \emph{whether} to create skills. Claude exhibits efficiency-maximizing behavior: it autonomously evaluates whether the abstraction overhead is justified before committing to skill creation. In Trajectory A, Claude identifies that the easy task (9 API calls for 3 cat breeds) does not warrant skill abstraction and proceeds with direct calls, completing in 34 steps. In Trajectory C, facing a harder task (15 API calls for 5 cocktails), Claude creates a single skill that executes correctly 5 times with zero errors. In contrast, DeepSeek follows the system prompt more literally, attempting skill creation regardless of task complexity. In Trajectory B, it creates \texttt{process\_cat\_breed} for the same easy task despite minimal reuse benefit, and in Trajectory D, it persists through three failed skill creation attempts before abandoning the approach entirely.

\paragraph{Skill Creation Failures.}
DeepSeek's skill creation attempts reveal systematic issues. In Trajectory B, the created skill \texttt{process\_cat\_breed} is incomplete---its output schema omits \texttt{breed\_facts} and \texttt{breed\_encyclopedia} fields, requiring 8 additional repair operations. In Trajectory D, DeepSeek attempts skill creation three times (\texttt{process\_cocktail}, \texttt{process\_cocktail\_v2}, \texttt{process\_cocktail\_v3}), each failing with syntax errors such as ``unexpected token'' and ``return is invalid outside function.'' These errors indicate that DeepSeek treats skill creation as template expansion rather than program synthesis.

\paragraph{Skill Execution Failures.}
Even when skills are successfully saved, execution failures reveal deeper issues. In Trajectory B, all three \texttt{execute\_skill} calls produce incomplete results with warnings about missing fields. The skill's internal logic failed to properly chain the three required API calls. In Trajectory D, the \texttt{execute\_skill} call fails immediately with a runtime error, forcing the agent to fall back to manual API calls and ultimately failing the task.

\paragraph{Implications.}
These findings suggest that effective tool composition requires not just the \emph{ability} to create and execute skills, but the \emph{judgment} to know when abstraction is beneficial. The 5.3$\times$ token savings achieved by Claude in the hard task (213K vs. 1.14M tokens) compared to DeepSeek demonstrates that understanding-driven skill use leads to both higher success rates and greater efficiency.


\section{Prompt Templates}
\label{app:prompts}

This section presents the prompt templates used in our experiments, including the system prompt for skill-enabled modes and representative task prompts across different difficulty levels.

\subsection{System Prompt for Skill Reuse}
\label{app:system_prompt}

In skill mode, agents receive an augmented system prompt that introduces the skill abstraction mechanism. The prompt provides: (1) available skill tools (\texttt{save\_skill} and \texttt{execute\_skill}); (2) guidelines for when to create skills; (3) script authoring rules; and (4) a concrete example demonstrating the skill creation and execution workflow.

The key design principle is \emph{minimal intervention}: rather than prescribing when agents should use skills, we provide the capability and let agents autonomously decide based on task structure. This enables fair comparison between skill-enabled and baseline modes, as the core task instructions remain identical.

\begin{SystemPromptBox}{System Prompt: Skill Reuse Mode}

\SPLabel{Skill Tools:}
You have access to skill cache tools to save and execute reusable scripts:
\begin{itemize}
  \setlength{\itemsep}{2pt}
  \setlength{\parskip}{0pt}
  \setlength{\parsep}{0pt}
  \item \texttt{save\_skill} --- Save an executable script as a reusable skill
  \item \texttt{execute\_skill} --- Execute a saved skill with different arguments
\end{itemize}

\SPDivider

\SPLabel{When to Use:}
For repetitive operations (processing multiple items, files, etc.), create a skill to encapsulate the workflow, then execute it for all items. You can create skills based on tool schemas without calling the tool first---especially efficient when tools return large data.

\SPDivider

\SPLabel{Script Rules:}
\begin{enumerate}
  \setlength{\itemsep}{2pt}
  \setlength{\parskip}{0pt}
  \setlength{\parsep}{0pt}
  \item Use \texttt{call\_tool()} for ALL tool calls: \texttt{call\_tool('tool\_name', arg1=val1, ...)}
  \item \texttt{call\_tool()} returns DIRECT result---use it directly without \texttt{.get("result")} wrapper
  \item MUST set \texttt{result} variable---this is what gets returned from \texttt{execute\_skill}
  \item Modules available: \texttt{re}, \texttt{json}, \texttt{os} are pre-imported
  \item No recursion: Cannot call skill tools within skills
\end{enumerate}

\SPDivider

\SPLabel{Example:}
{\small
\begin{verbatim}
save_skill({
  "skill_name": "analyze_project",
  "script_code": ...
})

execute_skill({
    "skill_name": "analyze_project"
    "args": {"path": "org/repo1"}}
\end{verbatim}
}

\SPDivider

\SPLabel{Best Practices:}
\begin{itemize}
  \setlength{\itemsep}{2pt}
  \setlength{\parskip}{0pt}
  \setlength{\parsep}{0pt}
  \item \textbf{Token Efficiency:} Extract only fields needed for final output
  \item \textbf{Maximize ROI:} Create skill early, execute for ALL items (beneficial when $N \geq 3$)
  \item \textbf{Fallback:} If skill fails 2--3 times, process items directly
\end{itemize}

\end{SystemPromptBox}


\subsection{Task Prompt Examples}
\label{app:task_prompts}

Task prompts describe the objective, required outputs, and available domain-specific tools. We present three representative examples from our scaled task suite, spanning easy (E), medium (M), and hard (H) difficulty levels. The scaling follows a systematic pattern: easy tasks involve $3 \times 3 = 9$ API calls, medium tasks involve $4 \times 4 = 16$ calls, and hard tasks involve $5 \times 5 = 25$ calls.

Each prompt specifies:
\begin{itemize}
    \item \textbf{Objective}: The data collection or analysis goal
    \item \textbf{Output format}: JSON schema for structured results
    \item \textbf{Available tools}: Domain-specific APIs (prefixes removed for clarity)
    \item \textbf{Scale}: Number of subtasks and API calls per subtask
\end{itemize}

Note that skill-related tools (\texttt{save\_skill}, \texttt{execute\_skill}) are \emph{not} mentioned in task prompts---they are injected via the system prompt only in skill-enabled modes. This ensures that baseline (Normal) mode and skill-enabled modes receive identical task instructions.

% Task Prompt Examples for Scaled Tasks
% Requires: \input{task_prompt_template.tex} in preamble

% =========================================================================
% Task 1: Easy Difficulty - Cat Facts Collector (E1)
% =========================================================================

\begin{TaskPromptBox}{cat-facts-collector/e1 ~~{\small[Easy]}}

\TPLabel{Prompt:}
Create encyclopedia entries for \textbf{3 cat breeds} (Persian, Siamese, Maine Coon) using 3 API endpoints per breed. For each breed, collect: (1) \textit{Breed Profile} --- basic info and characteristics; (2) \textit{Country Relatives} --- breeds from same country; (3) \textit{Coat Family} --- breeds with similar coat. Compile a summary with statistics across all breeds and save results to cat\_encyclopedia.json.

\TPDivider

\TPLabel{Available tools:}
\begin{itemize}
  \setlength{\itemsep}{4pt}
  \setlength{\parskip}{0pt}
  \setlength{\parsep}{0pt}
  \item breed\_profile(breed\_name)\\
        {\small Get breed info and characteristics}
  \item breed\_relatives(country)\\
        {\small List breeds from same country}
  \item breed\_coat\_family(coat\_type)\\
        {\small List breeds with similar coat}
  \item write\_file(path, content)\\
        {\small Save JSON output}
  \item claim\_done(status)\\
        {\small Signal task completion}
\end{itemize}

\TPDivider

{\small\color{gray} \textbf{Scale:} 3 subtasks $\times$ 3 API calls = 9 total calls}

\end{TaskPromptBox}

\vspace{4mm}

% =========================================================================
% Task 2: Medium Difficulty - Cocktail Menu Generator (M1)
% =========================================================================

\begin{TaskPromptBox}{cocktail-menu-generator/m1 ~~{\small[Medium]}}

\TPLabel{Prompt:}
Create a cocktail menu for \textbf{4 classic cocktails} (Margarita, Mojito, Old Fashioned, Martini) using 4 API endpoints per cocktail. For each cocktail, collect: (1) \textit{Search} --- find cocktail by name; (2) \textit{Details} --- full recipe and instructions; (3) \textit{By Ingredient} --- list cocktails using main ingredient; (4) \textit{By Category} --- list cocktails in same category. Calculate complexity rating (Easy/Medium/Complex based on ingredient count) and estimated prep time. Save results to cocktail\_menu.json.

\TPDivider

\TPLabel{Available tools:}
\begin{itemize}
  \setlength{\itemsep}{4pt}
  \setlength{\parskip}{0pt}
  \setlength{\parsep}{0pt}
  \item search(name)\\
        {\small Search cocktail by name}
  \item details(id)\\
        {\small Get full recipe and instructions}
  \item by\_ingredient(ingredient)\\
        {\small List cocktails with ingredient}
  \item by\_category(category)\\
        {\small List cocktails in category}
  \item write\_file(path, content)\\
        {\small Save JSON output}
  \item claim\_done(status)\\
        {\small Signal task completion}
\end{itemize}

\TPDivider

{\small\color{gray} \textbf{Scale:} 4 subtasks $\times$ 4 API calls = 16 total calls}

\end{TaskPromptBox}

\vspace{4mm}

% =========================================================================
% Task 3: Hard Difficulty - GitLab Deep Analysis (H1)
% =========================================================================

\begin{TaskPromptBox}{gitlab-deep-analysis/h1 ~~{\small[Hard]}}

\TPLabel{Prompt:}
Perform a comprehensive analysis of \textbf{5 GitLab repositories} (gitlab-runner, gitaly, gitlab-pages, gitlab-shell, cli). For each project, collect: (1) \textit{Project Info} --- stars, forks, description; (2) \textit{Contributors} --- top 5 by commit count; (3) \textit{Recent Commits} --- last 20 commits with authors; (4) \textit{Branches} --- all branches with protection status; (5) \textit{Issues} --- open count and recent titles. Calculate activity score (0--100) based on commits (40\%), contributors (30\%), issues (20\%), branches (10\%). Determine health status: \textit{healthy} ($\geq$70), \textit{moderate} (40--70), \textit{inactive} ($<$40). Save results to gitlab\_analysis\_results.json.

\TPDivider

\TPLabel{Available tools:}
\begin{itemize}
  \setlength{\itemsep}{4pt}
  \setlength{\parskip}{0pt}
  \setlength{\parsep}{0pt}
  \item get\_project\_info(project\_path)\\
        {\small Get project details (stars, forks, description)}
  \item get\_contributors(project\_path)\\
        {\small Get contributor list}
  \item get\_commits(project\_path, limit)\\
        {\small Get commit history}
  \item get\_branches(project\_path)\\
        {\small Get branch information}
  \item get\_issues(project\_path)\\
        {\small Get issue list}
  \item write\_file(path, content)\\
        {\small Save JSON output}
  \item claim\_done(status)\\
        {\small Signal task completion}
\end{itemize}

\TPDivider

{\small\color{gray} \textbf{Scale:} 5 subtasks $\times$ 5 API calls = 25 total calls}

\end{TaskPromptBox}